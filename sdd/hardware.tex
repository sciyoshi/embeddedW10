\chapter{Hardware}

\section{Controller: Microchip PIC32MX695F512H}
The core of the system is the powerful Microchip PIC32MX695F512H controller, which directly interfaces with each of the system's peripherals. This controller was chosen due to the high processing speed, low cost, and abundant peripherals. It operates at 3.3V with an instruction clock of 80MHz, referenced to an 8MHz crystal for improved stability with the extreme temperature fluctuations provided by the amplifiers. The controller is based on a MIPS32 M4K 32-bit core, featuring a 5-stage pipeline, 512kB of flash program memory (with an additional 12kB available for a bootloader), and 128kB of RAM.
\subsection{Hardware Peripheral Support}
The PIC32MX695F512H Its abundant hardware peripherals include:
\begin{itemize}
\item 4 I\superscript{2}C channels
\item 3 SPI channels
\item 16 channel 10-bit ADC
\item 10/100Mbit ethernet controller
\item USB 2.0 transciever with On-The-Go (OTG) and full-speed capabilities
\item 6 UART channels
\item 5 16-bit timers
\item 5 Output Compare modules for PWM output
\item 5 Input Capture modules for PWM input
\item Up to 53 general purpose I/Os, many of which are 5V tolerant. 
\end{itemize}

The following peripherals are used in our system:
\subsubsection*{I\superscript{2}C}
I\superscript{2}C (\textbf{I}nter-\textbf{I}ntegrated \textbf{C}ircuit) is a bi-directional bus which requires minimal connectivity and can have any number of masters or slaves. Only two signalling wires are required: SDA (\textbf{S}erial \textbf{DA}ta) and SCL (\textbf{S}erial \textbf{CL}ock.) Since, unlike with SPI, there is no \textit{select} line, slave selection is done by prefixing all operations with the address of the intended slave. The I\superscript{2}C bus is used primarily as a messaging bus due to its minial connectivity requirements and low maximum clock rate, which is usually 100 or 400kHz, making it ill suited for applications requiring high data throughput. All of our peripherals operate only as slaves and the controller acts only as a master.
\subsubsection{10-bit ADC}
The 10-bit \textbf{S}uccessive \textbf{A}pproximation \textbf{R}egister (SAR) \textbf{A}nalog-to-\textbf{D}igital \textbf{C}onverter (ADC) has a 16-input analog multiplexer and can sample values from 0V to 3.3V (Vcc) at up to 1MHz.

\subsection{Programming}
Microchip offers a free C compiler, MPLAB C32, for educational use. Compiled code can be loaded to the chip using the \textbf{I}n-\textbf{C}ircuit \textbf{S}erial \textbf{P}rogramming (ICSP) header on the board and a PICKit2, which is a small USB programming device. Programming may also be done directly over USB to the controller by incorporating Microchip's USB bootloader into our code.

\section{Amplifier A -- Cirrus CS4245}

\section{Amplifier B -- Cirrus CS4412}

\section{EEPROM -- Microchip 24FC1025}

\section{LCD -- Hitachi HD44780}
The 16x2 character LCD employs the industry-standard protocol of the Hitachi HD44780 controller.

\section{Buttons}

\section{Continuous Rotary Switch}

\section{Power Regulation}

